\chapter{Introduction}
\label{chap:intro}
Since its invention as a network of few computers owned by universities and
research institutes in the US, the Internet has changed a lot. From a
relatively homogeneous network in terms of connectivity and hosts capabilities,
the Internet now allows very different peers, from mobile devices or objects to
powerful servers, to communicate through a variety of fixed and mobile
communication links. During this evolution, many different protocols and
network equipment was used but in the course of the time there was a
convergence to the IP protocol, not only for data communication but even for
real-time, multimedia, conversational communication services in both fixed and
mobile networks.

In the last years, the Internet has moved from an essentially fixed network to
a network accessed from mobile devices, using both Wi-Fi connections and data
connections like GPRS, Edge, 3G, and LTE (Long Term Evolution). That raised
also a number of new issues. First, mobile devices move and change location, so
it is more difficult to route Internet traffic routing. Beyond this issue, most
mobile devices can also jump from one access technology to another, such as
switching between Wi-Fi and LTE connectivity. Finally, with the deployment of
LTE networks, mobile networks traffic usage is becoming more and more
heterogeneous, as voice communication and multimedia streaming. This emphasize
the need for an appropriate quality of service, or at least a service level
that allows the real-time constraints or traffic prioritization policies for
different data types.

These requirements are highlighted even by a new emerging traffic typology, the
machine-to-machine communication. Nowadays the trend of smart objects make that
an increasing variety of different objects will be connected to the Internet.
These objects have usually different necessities and scarce resources in terms
of energy, computation power and bandwidth.

The heterogeneity into the traffic and the different Qualities of Service
required posed different challenges to the network of the future. Network
resource management and application-driven networking are suitable solution in
order to optimize packets that run through the Internet backbone networks and
allows application-specific enhancements. With the new 5G standards some of
these features will be deployed thanks even to the Service Function Chaining
technology. In this context network function, nowadays available only as a
combination of vendor-specific devices, will be software programs that can be
composed as ordered chains, to which packets must pass through. 

In this thesis is proposed a simple implementation of a Service Function Chain
inside a virtualized environment. The chain proposed exploit the capabilities
of Docker container and Kubernetes as container orchestrator. Also a minimal
implementation of software needed to manage and orchestrate the chains is
provided, to allow basic functionalities as creation, deployment and deletion. 


 \section{Document organization}
 
 This thesis is organized as follows:
 \begin{itemize}
  \item Chapter~\ref{chap:background} describes background information about 5G 
and the technologies used to build an ETSI compliant architecture
  \item Chapter~\ref{chap:prjan} explains the analysis performed on the 
requirements provided and the design choices performed.
  \item Chapter~\ref{chap:archimpl} provides an in-depth view of the 
high-level architectural implementation.
  \item Chapter~\ref{chap:conclusions} wraps up the overall project experience.
 \end{itemize}