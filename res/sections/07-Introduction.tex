\chapter{Introduction}
\label{chap:intro}
At the beginnings of the Internet was a network comprised of a few computers
owned by universities and research institutes. From a relatively homogeneous
network in terms of connectivity and hosts capabilities, the Internet now
interconnects a vast number of autonomous systems serving a wide range of
heterogeneous end-systems varying from mobile devices, things to powerful
computing platforms. All these entities interconnect and are able to
communicate through both fixed and mobile communication links in a converged
all-IP network. 

Nowadays connectivity is ubiquitous and taken for granted. The user has become
a hungry consumer as well as producer of data, the so called presume era. As a
consequence, mobile data traffic is surging and is expected to increase in the
near future. In this context, making things worse, are the new and emergent
verticals e.g., industry 4.0, massive IoT, vehicular etc., embodying different
requirements with different Quality of Service (QoS) and Quality of Experience 
(QoE) guarantees. 

It is becoming evident that over-provisioning in both access and core network
alone is not a sustainable strategy and cannot provide any value added
guarantees in the long-term. Network optimization and management capabilities
are becoming of paramount importance, providing the much flexibility to tune
and optimize for the services at hand. While up until now, network management
and optimization equated with installing middleboxes, running dedicated
software, in the end-to-end path between a user and a service, now, thanks to
the consolidation of virtualization technology new possibilities have emerged.
Indeed, virtualization provides the possibility to decouple the software
lifecycle from its hardware, providing the basis for a faster pace evolution
and innovation in the network side. The ETSI Network Function Virtualization 
(NFV) pursues this goal, providing a reference architecture framework leveraging
on virtualization, envisaging a centralized brain with end-to-end visibility
managing and orchestrating network resources.

In this context, application-driven networking is emerging as a viable
paradigm, providing the basis for application/service-specific enhancements. 
Service Function Chaining (SFC) is an emerging trend  in network design,
specifically defined for the support of application-driven-networking through an
ordered interconnection of service functions. All of the software entities
(e.g., load balancers, deep packet inspection, cloud communication, cloud
processing, etc.) involved can be designed as service functions and combined in
order to reach given efficiency, effectiveness, and flexibility gains compared
to the traditional middlebox approach. Removing dedicated hardware of from the
network will open the possibility to deploy new function in a really short span
of time as well as to update exiting ones. Finally, this emerging technology is
able to enhance the quality of end-to-end services: providing a disparate set of
chains for the different traffic typologies and enforcing the most suitable
policies to meet service requirements will leverage the overall quality of the
system. This concept, could be employed in synergy with container technology
orchestration frameworks, like Kubernetes, providing a basis for function
capability adaptation and scaling based on traffic load or
user demand.

In this thesis we propose an implementation of the Service Function Chain
reference architecture exploiting orchestration and virtualization capabilities.
The proposed approach exploits the capabilities of Docker container and
Kubernetes as container orchestrator. Furthermore, we described and developed a
proof-of-concept instrumentation solution capable of management and
orchestration of chain lifecycle.

\section*{Document organization}

This thesis is organized as follows:
\begin{itemize}
  \item Chapter~\ref{chap:background} describes background information about 5G 
and the technologies used to build an ETSI compliant architecture;
  \item Chapter~\ref{chap:rel_wk} discusses other work related to this topic;
  \item Chapter~\ref{chap:prjan} explains the analysis performed on the 
requirements provided and the design choices performed;
  \item Chapter~\ref{chap:archimpl} provides an high-level view of
  the architecture proposed;
  \item Chapter~\ref{chap:impl} describes in-depth the implementation;
  \item Chapter~\ref{chap:tests} evaluates the software developed, giving some
  metrics measured;
  \item Chapter~\ref{chap:future} discusses possible future enhancements;
  \item Chapter~\ref{chap:conclusions} wraps up the overall project experience.
\end{itemize}