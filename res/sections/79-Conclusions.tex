\chapter{Conclusion}
\label{chap:conclusions}

During this work we have studied the Service Function Chaining
technology and how components in this ecosystem are managed and orchestrated.
These concept were presented in this thesis together with the standards
associated to them and some proposal coming from the academic research.
Afterwards, we proposed our proof of concept solution that take advantage of
emerging technologies like Docker and Kubernetes, exploiting the
containerization of network functions.

After, we implemented our proposal and we deployed it in a real Kubernetes
cluster. The tests and metrics we retrieved reveal that this virtualized
approach is a feasible and reliable one, even if lots of challenges are still
open and even if our implementation is only a proof of concept. As we can see
from our minimal instrumentation developed, containers will allow, in the near
future, to deploy chains of network function in a quickly, scalable and fault
tolerant manner.

In conclusion, we can say that this approach to communication will open a wide
range of new opportunities for mobile and non communications, leveraging the
quality of all services provided through the Internet. This approach can include
even satellite communication, creating ad-hoc VNF and exploiting the properties
of these high-bandwidth links with long RTT, mitigating their issues. New
scenarios, such as IoT, Vehicular and Drone networks will be better supported,
deploying chain of function dedicated to those kind of traffic as well as to the
everyday usage as real-time connectivity or media streaming.

\section*{Personal considerations}
At the beginning of this work, I had not experience on most of these topics
or a really limited one. I took part in other project but never to one as large
as this. Playing with relatively new technologies as Kubernetes or with tools
developed for managing large infrastructures as Openstack made me think that
there are always something new to study and there are always room for
improvements. Also, other times that I have implemented some tools or some
programs I had nether had to thought that it will be encapsulated in a so much
wider and complex system as in this case, in which an incredible huge amount of
packets can, in the future, pass through.

Since the beginning we tried to work keeping this idea in mind and this lead us
to change different times approaches, choices and tools that we used. This
sometimes made me feel uncomfortable, because I thought that some of my
solutions or proposals can be suitable in a certain scenario, but it trained my
critical sensibility and it taught me to look for alternative approaches,
different from mine, both in the code and in general.