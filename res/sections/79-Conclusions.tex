\chapter{Conclusions}
\label{chap:conclusions}

During this work we have studied the Service Function Chaining
technology and how components in this ecosystem are managed and orchestrate.
These concept were presented in this thesis as well as standard associated and
some proposal coming from the academic research. After, we proposed our
proof of concept solution that take advantage of emerging technologies as Docker
and Kubernetes, exploiting the containerization of network functions.

After we implemented our proposal, deploying it in a real Kubernetes cluster.
Tests and metrics retrieved reveal that this virtualized approach is a feasible
and reliable approach, even if lots of challenges are still open and even if our
implementation is only a proof of concept. As we can see from our
minimal instrumentation developed containers will allow, in the near future, to
deploy chains of network function in a quickly, scalable and fault tolerant
manner. 

Concluding, we can say that this approach to the communication will open a wide
range of new opportunities for mobile and non communications, leveraging the
quality of all services provided through the Internet. This approach can include
even satellite communication, creating ad-hoc VNF and exploiting the properties
of these high-bandwidth links with long RTT, mitigating their issues. New
scenarios, such as IoT, Vehicular and Drone networks will be better supported,
deploying chain of function dedicated of those kind of traffic as well as
everyday usage as real-time connectivity or media streaming.