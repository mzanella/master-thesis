\chapter{Related Work}
\label{chap:rel_wk}



Since the SFC technology was proposed many work came out proposing different
implementation solutions. In this part there is an overview of some of them.

\cite{GhaznaviSAB16} proposed an optimization model to calculate the optimal
location of VNFs based on different metrics such as bandwidth, load balancing
and routes. This decomposes the SFC on performance requirements eliminating the
throughput bound of a chain to a single VNF or a single physical machine.
Mathematical models developed was tested in a simulation environment. A critical
enhancement to this work could be to test algorithm in a real environment, under
different conditions of traffic load: if it efficient even in this scenario it
can be useful to the orchestration and management part of the ecosystem because
it can reduce resource waste.

Other implementation tested in a simulated environment
are~\cite{csoma2014escape} and~\cite{kim2016evaluations}. In both of them the
network functions are deployed on Mininet. The former work provides a full
implementation of the SFC environment, including SFC configuration, allocating
VNFs into the physical resources, flow routing across the VNFs based on
policies, and provisioning live management information on operating VNF
instances, but no performance evaluation of the work is presented. In the latter
case chains are identified by VLAN tag and the VNF chaining is created
exploiting only using Floodlight controllers.

Another noteworthy work is~\cite{xia2015optical} that discuss the packet
delivery through function chaining in high capacity networks exploiting
aggregate flow steering. The architecture proposed makes use of SDN and cloud
platforms to deploy VMs that will host VNFs. Assumption are made about the SDN
network elements: connection through optical circuit switching and use of Wave
Selective Switching (WSS) to route (switch) signals between optical fibers on a
per-wavelength basis. Thanks to the implementation of the SDN controller and the
communication between the previous element and the VNFM, this architecture is
able to provide good capabilities in terms of flexibility. Another element that
helps the flexibility is the high performance link: in this way they get rid of
problems in terms of connection time and bandwidth. However, in the paper
authors assumes really specific hardware in the underlying infrastructure that
connects the overall system: this is not suitable to be deployed in a general
purpose infrastructure or using IaaS. It will be interesting to evaluate the
same work on different instrumentation. A clashing aspect is the time to make a
VM up and running compared to times of service chaining configuration/packet
delivery. Using containers the instantiation of virtual functions should employ
less time.

A relevant work on function chaining in network is~\cite{trajkovska2017sdn}. In
this work author efforts are concentrated on joining SDN and VNF technologies to
create SFCs. They developed an SDN-driven SDK, a virtual traffic classifier to
analyze traffic in real-time and decide the optimal chain for it and a VNF
specialized on video transdecoding, used event to evaluate the performances of
the whole system. Chaining is performed exploiting layer 2 connectivity, using
OpenvSwitch as virtual network switches an OpenDayLight in order to route the
traffic on a set of policies. However the solution does not provide any
mechanism to perform an automatic deployment of chains and services: this can be
an issue in particular in case of VNF fault. Orchestration and some mechanisms
of fault tolerance are critical in particular in a real environment, enabling
autoscaling on user demand or traffic load. Automated deployment also make
possible the recreation of a virtual machine in case of fault. As a matter of
fact that a virtual function can fail depending on an erroneous load balancing
policy or too heavy computational load and it is critical in order to do not
interrupt end-to-end communication to bring up the VNF as soon as possible. This
recreation process can be slow due to the usage of VMs to deploy the functions,
that could take minutes to startup.

The paper~\cite{kriti2017dnfc} introduce the possibility using container in SFC
deployment. The chain of container is created exploiting Openflow to create flow
to route traffic to the containers, that have to be installed on the host
machine. This work however does present any form of management or orchestration
of the chain, neither an automatic deployment. Moreover it requires installation
of part of the system on the host machine: in this way there is not a fully
virtualized environment. The former issue has some implication: the VNF has a
strong relation on the host on which is running and the system could not be
simply be moved from one machine to another as it could be done only using
containers. Finally this paper does not provide any form of system evaluation
neither in terms of performances nor in terms of resilience. Fault recovering
and autoscaling could be achieved only improving the system introducing other
technologies, such as Docker Compose, Kubernetes or Docker Swarm.






