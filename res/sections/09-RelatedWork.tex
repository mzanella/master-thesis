\chapter{Related Work}
\label{chap:rel_wk}

In this chapter we present some relevant researches related to topics covered
in this thesis. 

\section{Network Function Virtualization}
After a through survey of literature material we found several research works
and actual ETSI-NFV implementations.

An example is the one described in~\cite{nogales2018nfv} in
which the ETSI MANO architecture is presented on top of a network of drones.
This opens a wide range of possibilities, allowing to provide connectivity even
in areas where there is no infrastructure or it is momentarily not
available, for example after a natural disaster. Even if this solution was
really flexible and quick to deploy it would need a VIM infrastructure to which
be connected.

In~\cite{yu2015network}, authors are instead oriented towards the study of the
opportunity to deploy the overall system on the top of multi-tenant cloud
architectures. Challenges in this context concern mainly the possibility to
enforce requirements, such as policies and performance. Moreover, demanding the
resource management to the cloud infrastructure means that it must be able to
optimize the resource utilization.

A significant proposal was presented in~\cite{cziva2017container}, one
of the first papers that evaluate the possibility to use container technology on
the network edges. This work is based on~\cite{cziva2015container} in which the
authors define an architecture that orchestrates containers, and the components
of which can be mapped on ETSI MANO architectural definition. The usage of
containers enable the possibility of deployment on less powerful devices, while
putting them on network edge opens a wide range of opportunities to classify
packets and treat them in an adequate manner as soon as possible. Another paper
that proposes the use of containers for VNF is~\cite{hoang2018extended}, in
which function are deployed both on Openstack and on Kubernetes exploited as
VIMs, using TOSCA definition to describe containers and Tacker for the
management as the network function orchestrator, even if they do not provide an
evaluation of the solution. Another aspect to consider is that they do not
mention the possibility to compose their function. Finally, as stated by the
authors, standard does not fully support features for container by now.


\section{Network Slicing and Provisioning}
Network slicing is one of the most important technology improvements that allows
to move towards the 5G technology. In~\cite{chartsias2017sdn} a
network slicing approach based on SDN in multi-tenancy scenarios is presented.
The implemented testbed  proposes a centralized approach to manage physical and
virtual resources in different slices. In~\cite{peuster2016medicine} instead,
the authors describe a NFV-based platform, that allows management and
orchestration showing the importance of virtual function in the network slicing
context.

This emerging technology open a wide number of new challenges. An example is the
possibility to ensure proper security on communication, hard also due
to the concept of shared services, that can not provide suitable isolation. 
In~\cite{kotulski2017end} the authors describe possible security issues, stating
that the isolation required may vary depending on usage scenarios. Even the ETSI
MANO system considers isolation only in terms of performance level. Other
challenges are highlighted in~\cite{li2017network}, in which in addition to
isolation and security concerns, it is pointed out possible problems of resource
allocation and sharing algorithms and that some virtualization methods, used
in wired context, cannot be applied in wireless counterpart.

\section{SFC Implementations}

Since the SFC technology was first proposed, many works came out offering
different implementation~\cite{medhat2017service} solutions. Some of them are
presented in the following overview.

The work~\cite{GhaznaviSAB16} proposed an optimization model to calculate the
optimal location of VNFs, based on different metrics such as bandwidth, load
balancing and routes. This decomposes the SFC in performance requirements,
eliminating the throughput bound of a chain to a single VNF or a single physical
machine. Mathematical models developed were tested in a simulation environment. A
critical enhancement to this work could be achieved by testing the algorithm in
a real environment, under different conditions of traffic load: if it results
efficient even in this scenario it could be useful to the orchestration and
management part of the ecosystem because it could reduce resource waste.

\cite{csoma2014escape} and~\cite{kim2016evaluations} are other implementations
that were tested in a simulated environment. In both of them the
network functions are deployed on Mininet. The former work provides a full
implementation of the SFC environment, including SFC configuration, allocating
VNFs into the physical resources, flow routing across the VNFs based on
policies, and provisioning live management information on operating VNF
instances, but no performance evaluation of the work is presented. In the latter
case chains are identified by VLAN tag and the VNF chaining is created
exploiting only using Floodlight controllers.

A significant differentiation on SFC implementation proposal is whether a
combination of NFV and SDN or if they are not adopted together. If no
orchestration is available, SDN control plane is in charge to do so. Whether
both of them are involved, different solutions are suggested. Another criteria
that differentiate solutions is the SFC forwarding plane that could make use of
NSH or not. Moreover some implementations, like~\cite{soares2015toward}
and~\cite{abujoda2015midas} does not allow SFC updating, deploying only static
chains.

Another noteworthy work is~\cite{xia2015optical} which discusses packet
delivery through function chaining in high capacity networks exploiting
aggregate flow steering. The architecture proposed uses SDN and cloud
platforms to deploy VMs that will host VNFs. Assumption are made about the SDN
network elements: connection through optical circuit switching and use of Wave
Selective Switching (WSS) to route (switch) signals between optical fibers on a
per-wavelength basis. Thanks to the implementation of the SDN controller and the
communication between the previous element and the VNFM, this architecture is
able to provide good capabilities in terms of flexibility. Another element that
helps with flexibility is the high performance link: in this way they are gotten
rid of problems in terms of connection time and bandwidth. However, in the
paper, the  authors ask for a really specific hardware in the underlying
infrastructure that connects the overall system: this is not suitable to be
deployed in a general purpose infrastructure or using IaaS. It would be
interesting to evaluate the same work on different instrumentation. A clashing
aspect lays in the time to make a VM up and running, compared to times of
service chaining configuration/packet delivery. Using containers, the
instantiation of virtual functions should employ less time.

A relevant work on function chaining in network is~\cite{trajkovska2017sdn}.
Here the authors' efforts are concentrated on joining SDN and VNF technologies
to create SFCs. They developed an SDN-driven SDK, a virtual traffic classifier
to analyze traffic in real-time and decide the optimal chain for it and a VNF
specialized on video transdecoding, used even to evaluate the performances of
the whole system. Chaining is performed therefore exploiting layer 2
connectivity, using OpenvSwitch as virtual network switches an OpenDayLight in
order to route the
traffic on a set of policies. However the solution does not provide any
mechanism to perform an automatic deployment of chains and services, which can
be an issue in particular in case of VNF fault. Orchestration and some
mechanisms of fault tolerance are critical in a real environment, because they
enable autoscaling on user demand or traffic load. Automated deployment also
makes the recreation of a virtual machine in case of fault possible. As a matter
of fact that a virtual function can fail depending on an erroneous load
balancing policy or too heavy computational load and it is critical to bring up
the VNF as soon as possible in order to do not interrupt end-to-end
communication. This recreation process can be slowed down because of the usage
of VMs to deploy the functions, which could take minutes to startup.

The paper~\cite{kriti2017dnfc} introduces the possibility of using containers in
SFC deployment. The chain of containers is created exploiting Openflow to create
flows that route traffic to the containers, that has to be installed on the host
machine. This research however, does not present any form of management or
orchestration of the chain, neither an automatic deployment. Moreover, it
requires the installation of part of the system on the host machine: in this way
there is not a fully virtualized environment. The former issue has some
implication: the VNF has a strong relation on the host on which is running and
the system could not be simply be moved from one machine to another as it could
be using a fully container-based approach. Finally this paper does not provide
any form of system evaluation neither in terms of performances nor in terms of
resilience. Fault recovering and autoscaling could be achieved only by improving
the system introducing other technologies, such as Docker Compose, Kubernetes or
Docker Swarm.

To the best of our knowledge, all general purposes solutions rely on virtual
machines, lowering system performance due to overhead caused by hypervisor-based
virtualization, or only provides a static, not fully virtualized solution for
SFCs. Moreover, there are not suitable works that integrates Kubernetes to
orchestrate and handle them.

