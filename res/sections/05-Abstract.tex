%**************************************************************
% Abstract
%**************************************************************
\cleardoublepage
\phantomsection

\vspace*{\fill}

\vspace*{-5cm}

\thispagestyle{empty}

\par
\begingroup
\leftskip4em
\rightskip\leftskip
\section*{\centering Abstract}
Nowadays, traffic on Internet is becoming more and more heterogeneous and
different types of devices are capable to access it to pursue different goals.
Networks and infrastructures must follow these changes and be enaught flexible
for future traffic evolutions. Adapting networks on traffic requirements will
make possible to perform different optimizations, augmenting the overall system
efficiency and providing better services to the end user. A suitable solution is
the Service Function Chaining. This approach allows the composition of different
network functions based on typology of traffic and Quality of Service
requirements. 

In this thesis we present an experimental implementation of a Service Function
Chaining framework that exploits the lightweight approach of container
technology to create and manage network functions.
\par\endgroup
\vspace*{\fill}