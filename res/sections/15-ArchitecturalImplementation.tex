\chapter{A two-layer service orchestration architecture}
\label{chap:archimpl}
 
In this chapter is discussed the architecture of the proposed solution.
We decided to design a two-layer MANO architecture that will manage
the service function chains. In the following we provide a brief overview of its functional components along with some deployment considerations exploiting the Kubernetes orchestration framework.

\section{A top-down Overview}
\begin{figure}[H]
  \centering
  \includegraphics[width=\textwidth]{Architecture}
  \caption{Architecture of a two-layer orchestration framework}
  \label{chap:archimpl:img:architecture}
\end{figure}
In Figure~\ref{chap:archimpl:img:architecture} is depicted the overall
architecture of the expected implementation. All components communicate with
each other thanks to RabbitMQ and the system is orchestrated by Openbaton. Our
proposal is a refinement of the ETSI NFV framework in which the technologies
involved are defined. The low-level orchestration handles
repositories for VNFs and SFCs definitions, monitoring and service deployment.

The higher level, instead, instructs the lower one through the APIs that it
exposes, configuring and managing services. 

After this exercise, we performed a technology review and several issues emerged
In fact, we choose to deploy our
platform on Kubernetes but, given time and resources, we were not able to make
Openbaton and this container orchestrator work together.

Openbaton has a plugin structure that allows developer to integrate new
features. We tried to developed an Openbaton VIM driver to allow this MANO to
deploy functions and chains on Kubernetes, but it requires to manage their
definitions using TOSCA. This standard language for describing cloud based
services topology focus on hypervisor virtualization and its resource usage.
This was an issue to our problem in which we have to exploit containers.

\section{Our proposal}
After evaluating the technologies and the starting architecture, we decided to
implement our management and orchestration solution, to be integrated with
Kubernetes and container technologies, \texttt{Harbor}, our MANO proposal, manages data
repositories for VNFs and SFCs definition as well as their lifecycle.
It was implemented to expose a RESTful API that allow managing
component deployed in a Openbaton-like manner. 

\begin{figure}[H]
  \centering
  \includegraphics[width=\textwidth]{architectureproposal}
  \caption{Final architecture}
  \label{chap:archimple:sec:secondattempt:img:attempt2v1harbor}
\end{figure}

To describe SFC, \texttt{Harbor} uses a JSON definition, for VNF it requires
instead a YAML definition, as the one used by Kubernetes. 

Thanks to the APIs, it is possible to integrate our system with Openbaton: in
fact it is possible to create a plugin and map it call to \texttt{Harbor} ones.
In order to fully support this tool the real effort will be to provide the
possibility to deploy chains and virtual functions on Kubernetes. 

To Kubernetes is demanded the management of the connectivity among the
components of the platform and the handling of the virtualized resources.
Nodes on Kubernetes are divided in master and slaves: the former gives access to
Kubernetes APIs, allowing for example to launch containers on the cluster, and
orchestrate other nodes, instead the latter elements, also called
\emph{Minions}, are the nodes on which Pods are deployed.

Kubernetes default installation is comprehensive of a DNS, \emph{CoreDNS}, for
name resolution. In order to reduce the number of requests to the DNS on each
Minion is deployed with a Dnsmasq instance, a local DNS cache that cooperate
with CoreDNS.