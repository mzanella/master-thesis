\chapter{Future work}
\label{chap:future}

Some further enhancements are needed to solve the problems discussed in
Section~\ref{chap:impl:sec:problems}. Some of these issues could be addressed
by adopting specialized libraries capable of handling socket I/O in a more
efficient manner and due to time constraint. Preserving the semantics
of the components, more efficient implementations are needed to use this
approach in a real environment, taking more care of resource usages. Another
important enhancement is to make \harbor{} able to provide scaling and
autoscaling capabilities to components deployed, in order to exploit all the
features given by Kubernetes. Moreover it can be integrated with Openbaton. In
fact, during the study of the available technologies on the market we focused
on this MANO implementation. It allows developers to expand the solution with
drivers and plugins to support different VIM and VNFM (so different VNF) types.
Thanks to \harbor{} APIs it will be possible to develop an Openbaton component
whose only aim is to call \harbor{} demanding the responsibility to manage a
certain request.

\harbor{} can also be enhanced improving the chain update system. At the current
stage all the running chains are updated immediately, but it can be useful in
some contexts adopt a less brutal approach can be a better solution, leaving
the communication that make use of the old version of the chain to terminate
and making new connections to use the new definition.

Packet transmission protocol used to exchange data inside the platform is UDP.
It is lighter than TCP but it does not manage packet loss and applications have
to manage these cases. In the implementation presented there are no mechanism of
packet recovering that must be implemented. Further other protocol can be use to
improve performance. For example, in 2012 Google presented QUIC, an UDP-based
transport layer network protocol that has the aim to improve performances of
current transport layer protocols.

During technologies evaluation we choose Kubernetes as VIM instead of Docker
Swarm. An interesting future work can evaluate this alternative container
orchestrator, comparing performances and implementing, if needed, different
solutions. Also making different Kubernetes instances communicate creating
a Federation will be a significant enhancement. At the time we developed the
project this feature was not tested because was not stable yet.

Another component that must be implemented is the \texttt{Classifier}. This
component can exploit deep packet inspection techniques to both make decisions
on the chain to be traversed from packets but even to make possible to
detect some sort of attacks as denial of service attacks: discarding packets
before packets arrive to the final destination can improve the end-to-end
communications security. Also some machine learning capabilities can be added
to it. Retrieving metrics from the MANO about the communications and
transmission can help the \texttt{Classifier} to make some decisions.

Tests conducted show that the systems work and it is capable to recover in case
of a VNF fault in case of light traffic. A further improvement will analyze the
behavior of the system in case of huge traffic with and without scaling
features. If the solution is still responsive even under a large traffic load
implies that the combination of Kubernetes and SFC is a suitable scenario to
which move network functions.

Concluding, it is important to point out that some security aspects were
neglected. This is acceptable, also considering the nature and objective of our
work which was that of providing a proof of concept implementation of an
experimental technology. Components as MANO and the VNFs must be secured to
make not possible to an intruder to take their control and hijack
communications.